\documentclass[12pt]{amsart}

\addtolength{\hoffset}{-2.25cm}
\addtolength{\textwidth}{4.5cm}
\addtolength{\voffset}{-2.5cm}
\addtolength{\textheight}{5cm}
\setlength{\parskip}{0pt}
\setlength{\parindent}{15pt}
\usepackage{amsthm}
\usepackage{amsmath}
\usepackage{amssymb}
\usepackage[colorlinks = true, linkcolor = black, citecolor = black, final]{hyperref}
\hypersetup{urlcolor=black}
\usepackage{graphicx}
\usepackage{multicol}
\usepackage{ marvosym }
\usepackage{wasysym}
\usepackage{tikz}
\usepackage{enumitem}
\usepackage{scrextend}
\usetikzlibrary{patterns}

\newcommand{\ds}{\displaystyle}

\setlength{\parindent}{0in}

\pagestyle{empty}

% ----------------------------

% The "stuff" above here is called the preamble of the document.  It sets the margins and loads special packages.  Probably the only reason you would need to edit something above here would be to add a package to do something very specific... but probably everything you need is loaded already

% -----------------------------

\begin{document}

\thispagestyle{empty}

{\scshape 4BL - 003} \hfill {\scshape \large E\&M Lab for Sci/Eng} \hfill {\scshape Fall 2018}
 
\smallskip

\hrule

\bigskip

\textbf{\underline{Instructor:}} Alex Roberts\\
\textbf{\underline{Email:}}\\
\textbf{\underline{Slack Page:}} \url{http://4blworkspace.slack.com}\\
\textbf{\underline{Lab Textbook:}} “Physics 4BL Laboratory Manual”, CCSF, XanEdu (2013).\\
\textbf{\underline{Prerequisites:}} PHYC 4A, PHYC 4AL, MATH 110B\\
\textbf{\underline{Corequisites:}} PHYC 4B\\
\textbf{\underline{Advisories:}} MATH 110C\\
\textbf{\underline{Course Description:}} Second laboratory course in a calculus-based four semester sequence
covering the topics of electricity and magnetism.\\
\textbf{\underline{Major Learning Outcomes:}} \\ Upon completion of this course, a student will be able to:
\begin{enumerate}[label=(\Alph*)]
\item  Measure electromagnetism data and evaluate uncertainties with apparatus such as
multimeters, circuits, function generators, and oscilloscopes.
\item Report on procedural methods, evidence, uncertainties, and conclusions of electromagnetism laboratory exercises.
\item Assess and present relationships between results of real-world laboratory exercises
and electromagnetism theory.
\end{enumerate}
\textbf{\underline{Weekly Schedule:}}\\ 
Lab (4BL-003, CRN 77201): W 2:10pm-5:00pm in S158\\
Office Hours: Half an hour before class and on Slack.\\
\textbf{\underline{Important Dates:}}\\ 
Day Class Begins: \textbf{August 22, 2018}\\
Last Day to Drop with refund: \textbf{August 31, 2018}\\
Last Day to Add with instructor's approval: \textbf{September 7, 2018}\\
Last Day to Drop without a 'W' symbol: \textbf{September 7, 2018}\\
Last Day to Drop with a 'W' symbol: \textbf{November 8, 2018}\\
\textbf{\underline{Lab Procedure:}}\\ 
Each week, we will perform one lab from the lab textbook. \textit{Read the lab write-up in
the lab textbook in advance of arriving to lab.} \\ I will begin each lab with a brief lecture and demonstration of the lab. Students will then perform the lab. Generally, this involves performing the activities in the lab textbook (including answering all of the questions) and writing the results in your lab notebook. \textit{After you complete the lab, you must return the lab station to the state it was in before you arrived.}
This includes putting away any equipment that you took out \textit{exactly as you found it}.\\
Students are expected to keep a lab notebook for the course. All data, observations,
comments, analysis (including answers to any of the questions), and conclusions should be
written in \textit{blue or black ink} in the lab notebook.\\
All data acquisition should be completed during the three-hour lab period. Students must turn in their lab notebook when they leave. Students who stay the entire three-hour period will have the option of answering any additional questions asked in the lab textbook as well as writing the conclusion on their computers and sending me a direct message with the file attachment on Slack no later than Thursday at 2:00pm.\\
Most of the lab notebooks will be graded for completion only and, in aggregate, will
contribute 25\% to your final grade. Two of the lab notebooks will be graded for content
and, taken together, will contribute another 25\% to your final grade. Regardless of how a
particular lab is graded, students will be responsible for all material covered in that lab on the exams, and would benefit greatly in writing a good lab notebook for that lab.\\
\textbf{\underline{Exams:}}\\ 
There will be two exams during the semester. For each exam, you will be allowed to use
your lab notebook. No other written materials will be permitted, including the lab textbook.
The first exam will cover the experiments performed up to that point, and the second exam
will cover all experiments between the first exam and the second exam. Exam questions
may involve the theory behind the lab, details of the lab procedure, the functionality of the lab equipment (what it does and how it works), and data analysis. You may be called upon
to perform an experiment. \textit{The best preparation for the lab exam is to actively participate during the labs and to do the lab notebook.} Your lab partner(s) will be doing you no favor by doing all of the work for you.\\
\textbf{\underline{Attendance:}}\\
Attendance will be taken at the beginning of each lab. Attendance in and of itself will
not be counted towards you final grade - however, if you miss a lab, you will receive a
zero for that lab. The lowest lab score during the semester will be dropped, so students
can afford to miss one lab (although you will still be responsible for the material covered in that lab on the exam). Students are responsible for all material covered and announcements made during the lab whether they attend or not. \textit{Excessive absences will severely affect your grade, and may also be grounds for being dropped from the course.} There is also an FW grade which may apply to students who stop attending class after the withdrawal deadline and subsequently earn an F in the course. Earning an FW grade could have a negative impact on a student’s financial aid and/or immigration status.\\
\textbf{\underline{Canvas:}}\\
All enrolled students will be provided access to Canvas, which is the online learning
system available at CCSF. You may find some of the resources in the course page useful, including copies of handouts and forums.\\
\textbf{\underline{Final Grades:}}\\
The final lab grade will be based on the exams and the lab notebook in the following proportion:
\begin{center}
\begin{tabular}{l l}
Exams & 50\% (2x25\%)\\
Lab notebook (completion) & 25\%\\
Lab notebook (content) & 25\%\\
\hline
Total & 100\%
\end{tabular}
\end{center}
Assuming no curve, letter grades will be assigned as follows:
\begin{center}
\begin{tabular}{l l}
A & Above 90\%\\
B & 80\%-90\%\\
C & 65\%-80\%\\
D & 50\%-65\%\\
F/FW & below 50\%
\end{tabular}
\end{center}
\textbf{\underline{Policies:}}\\
Students are expected to arrive on time to lab, and to behave themselves while in the
lab. Proper behavior is especially critical in a laboratory, as the equipment you are using
can be damaged or can seriously hurt you or someone else if it is mistreated. Students are
expected to comply with all safety regulations.\\
Each student must turn in his or her own work. Naturally, it is to be expected that lab
partners will consult one another when performing the experiment and writing up the lab
notebook. However, \textit{each student’s lab notebook must be in his or her own words}.\\
Students found to have identical (or nearly identical) wording in their lab notebook will
end up sharing the grade for the lab notebook (i.e., the credit for \textit{one} lab notebook will
be divided equally among all students involved). Note that this penalty will affect the
student who is copied from as well as the student(s) who did the copying. In addition,
students found to have written responses copied from an external source (e.g., the internet,
lab textbook, a student enrolled in another lab section, etc.) will receive a zero for the lab notebook. Repeated violations may result in further disciplinary action according to the
CCSF Student Code of Conduct.\\
Exams are closed book and closed notes (except for the lab notebook). Students cannot
have any materials other than those necessary for taking the exam (pencils/pens, erasers,
calculator, rulers,...). Students are required to do their own work during the exams: there
is to be no communication of any kind, nor sharing of any materials during an exam. The use
of communication devices (such as cell phones and smart phones) is strictly prohibited, \textit{even
if used for non-communication purposes} (e.g., use of cell phone calculators is prohibited).
\textit{Students caught cheating on an exam will receive a zero, and may also be subject to further
disciplinary action according to the CCSF Student Code of Conduct.} Exam seating will be
assigned. I also reserve the right to check picture IDs before each exam.\\
\textbf{\underline{DSPS Statement:}}\\ \textbf{If you need course adaptations or accommodations because of a disability,
if you have emergency medical information to share with me, or if you need
special arrangements in case the building must be evacuated, please make an
appointment with me as soon as possible.}\\ You may also contact DSPS. They can be reached at: 
\begin{addmargin}[2em]{2em}
DSPS\\ Rosenberg Library --- R323 \\ 50 Phelan Avenue \\ San Francisco, California 94112 \\ (415) 452-5481 Voice \\ (415) 452-5451 TDD \\ (415) 452-5565 FAX \\ \url{http://www.ccsf.edu/en/student-services/student-counseling/dsps.html}
\end{addmargin}
\textbf{\underline{List of labs:}}
\begin{enumerate}
\item Multimeters
\item Resistance Characteristic Curves
\item DC Circuits
\item Cathode Ray Tube
\item Oscilloscope 1
\item Oscilloscope 2
\item Semiconductor Diode
\item RC Circuits
\item Magnetic Field Lab/Current Balance
\item Ratio of e/m for the electron
\item RCL Circuits
\item Motor Lab?
\end{enumerate}
\end{document}